\section{はじめに}

この文書は, ごく基本的なレポートや論文の例を示すものです. 
実際にこのソースを入力してタイプセット(コンパイル)し, 
タイトル, 著者名, 本文, 見出し, 箇条書きがどのように表示されるかを
確認してみましょう. 

\section{見出し}

この文書の先頭にはタイトル, 著者名, 日付が出力されています. 
特定の日付を指定することもできます. 

そして, セクションの見出しが出力されています. 
セクションの番号は自動的に付きます. 

\section{箇条書き}

以下は箇条書きの例です. これは番号を振らない箇条書きです. 

\begin{itemize}
  \item ちゃお
  \item りぼん
  \item なかよし
\end{itemize}

これは番号を振る箇条書きです. 

\begin{enumerate}
  \item 富士
  \item 鷹
  \item なすび
\end{enumerate}

\section{おわりに}

これは一段組の例ですが, 二段組に変更することもできます. 

解説文を読んで, このソースをいろいろと変更してみましょう. 
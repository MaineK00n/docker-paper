\documentclass[a4paper]{jsarticle}
\usepackage[ipa]{pxchfon}
\usepackage[dvipdfmx]{graphicx}
\usepackage{cite}
\usepackage[dvipdfmx,%
bookmarks=true,%
bookmarksnumbered=true,%
colorlinks=true,%
setpagesize=false,%
pdftitle={docker-paper},%
pdfauthor={MaineK00n},%
pdfsubject={paper},%
pdfkeywords={hyperref;}]{hyperref}
\usepackage{pxjahyper}

\begin{document}

\title{Docker-Papaer Sample}
\author{MaineK00n}
\maketitle

\section{はじめに}

この文書は,ごく基本的なレポートや論文の例を示すものです。
実際にこのソースを入力してタイプセット(コンパイル)し,
タイトル,著者名,本文,見出し,箇条書きがどのように表示されるかを
確認してみましょう。

\section{見出し}

この文書の先頭にはタイトル,著者名,日付が出力されています。
特定の日付を指定することもできます。

そして,セクションの見出しが出力されています。
セクションの番号は自動的に付きます。

\section{箇条書き}

以下は箇条書きの例です。これは番号を振らない箇条書きです。

\begin{itemize}
  \item ちゃお
  \item りぼん
  \item なかよし
\end{itemize}

これは番号を振る箇条書きです。

\begin{enumerate}
  \item 富士
  \item 鷹
  \item なすび
\end{enumerate}

\section{おわりに}

これは一段組の例ですが,二段組に変更することもできます。

解説文を読んで,このソースをいろいろと変更してみましょう。

\end{document}